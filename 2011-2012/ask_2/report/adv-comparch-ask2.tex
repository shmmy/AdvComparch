%{{{ preamble
\documentclass[a4paper,12pt]{article}
\usepackage{anysize}
\marginsize{2cm}{2cm}{1cm}{1cm}
%0.7\textwidth 6.0in \textheight = 664pt
\usepackage{xltxtra}
\usepackage{xunicode}
\usepackage{graphicx}
\usepackage{color}
\usepackage{xgreek}
\usepackage{fancyvrb}
\usepackage{minted}
\usepackage{listings}
\usepackage{enumitem} 
\usepackage{framed} 
\usepackage{relsize}
\usepackage{float} 
\usepackage{pstricks}
\usepackage{pst-node}
\usepackage{pst-blur}
\setmainfont[Mapping=tex-text]{FreeSerif}
%}}}
\begin{document}

\def\thesection {\roman{section}: }
\def\thesubsection {\Roman{subsection}) }

\include{title/title}

\section*{Εισαγωγή}

Στην άσκηση αυτή χρησιμοποιήσαμε ένα Framework για την προσομοίωση αλμάτων
όπως έχουν καταγραφεί από την εκτέλεση benchmarks από τη σουίτα SPEC2000. Τα
traces περιέχουν μόνο εντολές άλματος όπως αυτές πραγματοποιήθηκαν κατά την
εκτέλεση 100M εντολών.

%{{{ A1

\section*{Μελέτη των n-bit predictors}


\subsection*{A.1}
Σε αυτό το τμήμα μελετήσαμε την απόδοση των \{1..7\}-bit predictors χρησιμοποιώντας ως μετρική τα MPKI (Mispredictions Per
Thousand Instructions) με 16K BHT entries.


\begin{figure}[H]
	\centering
	\includegraphics[width=0.8\textwidth]{files/src-A1-results79-results_A1.png}
	\caption{1-7 bit predictors}
	\label{fig:A1}
\end{figure}


Όπως παρατηρούμε από το Σχήμα \ref{fig:A1}, βέλτιστος προβλέπτης για τα
περισσότερα benchmarks είναι ο 4-bit predictor καθώς εκείνος έχει το
χαμηλότερο missprediction rate χωρίς να έχει πολύ μεγάλh πολυπλοκότητα στο
hardware.


%}}}

\pagebreak

%{{{ A2

\subsection*{A2}
Διαφοροποιούμε την υλοποίηση και μελετάμε \{1,2,4\}-bit predictors χρησιμοποιώντας ως μετρική τα MPKI (Mispredictions Per
Thousand Instructions) με μεταβλητό πλήθος BHT entries ώστε να έχουμε σταθερό
μέγεθος hardware και ίσο με 32Κ.

\begin{tabular}{r c r}
	HW & bits & BHT entries \\
	\hline
	\hline
	32K & 1 & 32K	\\
	32K & 2 & 16K	\\
	32K & 4 & 8K	\\
\end{tabular}

\begin{figure}[H]
	\centering
	\includegraphics[width=0.8\textwidth]{files/src-A2-results79-results_A2.png}
	\caption{1,2 και 4 bit predictors}
	\label{fig:A2}
\end{figure}

Και σε αυτή την περίπτωση που το hardware είναι περισσότερο (32K) καλύτερα
φαίνεται να συμπεριφέρεται ο 4 bit predictor, συνεπώς αυτόν θα επιλέγαμε και
σε αυτή τη φορά.

%}}}

%{{{ B1

\section*{B1. Μελέτη του BTB}

\begin{tabular}{l l}
	btb\_lines & btb\_assoc \\
	\hline
	\hline
	512K & 1 \\
	256K & 2 \\
	128K & 4 \\
	64K  & 8 \\

\end{tabular}

\begin{figure}[H]
	\centering
	\includegraphics[width=0.8\textwidth]{files/src-B1-results79-results_B1_direction.png}
	\caption{Direction misspredictions (direction MPKI)}
	\label{fig:B1a}
\end{figure}

\begin{figure}[H]
	\centering
	\includegraphics[width=0.8\textwidth]{files/src-B1-results79-results_B1_target.png}
	\caption{Target Misspredictions (target MPKI)}
	\label{fig:B1b}
\end{figure}


Παρατηρούμε πως το target missprediction παραμένει σταθερό σχεδόν, σε αντίθεση
με το direction missprediction που μειώνεται δραστικά στον 64x8 btb predictor.
Αυτή είναι και η επιθυμητή οργάνωση για τον btb.

%}}}
%
%{{{ C1

\section*{C1. Σύγκριση διαφορετικών predictors}
\begin{figure}[H]
	\centering
	\includegraphics[width=\textwidth]{files/src-C1-results79-results_C1.png}
	\caption{Σύγκριση predictors}
	\label{fig:C1}
\end{figure}

Από αυτούς καταλήγουμε πως καλύτερος είναι ο gshare predictor.

\subsection*{Tournament Hybrid predictors}

\begin{figure}[H]
	\centering
	\includegraphics[width=\textwidth]{files/src-C2-results79-results_C2.png}
	\caption{Σύγκριση hybrid predictors}
	\label{fig:C2}
\end{figure}

Από όλους τους hybrid predictors που χρησιμοποιήσαμε, καλύτεροι αποδείχτηκαν
οι gshare-globalhistory2 και gshare-globalhistory4 predictors. Τέλος
συγκρίνουμε τον hybrid gshare-globalhistory4 και gshare predictor.

\begin{figure}[H]
	\centering
	\includegraphics[width=\textwidth]{files/src-C3-results79-results_C3.png}
	\caption{Σύγκριση hybrid predictors}
	\label{fig:C3}
\end{figure}


Καταλήγουμε πως ο hybrid predictor που δημιουργήσαμε είναι ελαρφώς καλύτερος
από τον gshare predictor.

Ο πηγαίος κώδικας που χρησιμοποιήσαμε ήταν ο ακόλουθος:
\inputminted[linenos,fontsize=\scriptsize,frame=leftline]{cpp}{files/src-C1-predict.cpp}

\subsection*{Static Not-Taken}
\inputminted[linenos,fontsize=\scriptsize,frame=leftline]{cpp}{files/src-C1-nt_predictor.h}
\inputminted[linenos,fontsize=\scriptsize,frame=leftline]{cpp}{files/src-C1-nt_predictor.cpp}

\subsection*{Static Backward Taken Forward Not Taken}
\inputminted[linenos,fontsize=\scriptsize,frame=leftline]{cpp}{files/src-C1-btfnt_predictor.h}
\inputminted[linenos,fontsize=\scriptsize,frame=leftline]{cpp}{files/src-C1-btfnt_predictor.cpp}

\subsection*{4-bit predictor}
\inputminted[linenos,fontsize=\scriptsize,frame=leftline]{cpp}{files/src-C1-nbit_predictor.h}
\inputminted[linenos,fontsize=\scriptsize,frame=leftline]{cpp}{files/src-C1-nbit_predictor.cpp}

\subsection*{gshare predictor}
\inputminted[linenos,fontsize=\scriptsize,frame=leftline]{cpp}{files/src-C1-gshare_predictor.h}
\inputminted[linenos,fontsize=\scriptsize,frame=leftline]{cpp}{files/src-C1-gshare_predictor.cpp}

\subsection*{Local-History two-level predictors}
\inputminted[linenos,fontsize=\scriptsize,frame=leftline]{cpp}{files/src-C1-localhistory_predictor.h}
\inputminted[linenos,fontsize=\scriptsize,frame=leftline]{cpp}{files/src-C1-localhistory_predictor.cpp}

\subsection*{Global-History two-level predictors}
\inputminted[linenos,fontsize=\scriptsize,frame=leftline]{cpp}{files/src-C1-globalhistory_predictor.h}
\inputminted[linenos,fontsize=\scriptsize,frame=leftline]{cpp}{files/src-C1-globalhistory_predictor.cpp}


\inputminted[linenos,fontsize=\scriptsize,frame=leftline]{cpp}{files/src-C2-hybrid_predictor.h}
\inputminted[linenos,fontsize=\scriptsize,frame=leftline]{cpp}{files/src-C2-hybrid_predictor.cpp}

\subsubsection*{2-bit global history BHT=2}
\inputminted[linenos,fontsize=\scriptsize,frame=leftline]{cpp}{files/src-C2-hybrid_2bit_GH2_predictor.cpp}
\inputminted[linenos,fontsize=\scriptsize,frame=leftline]{cpp}{files/src-C2-hybrid_2bit_GH2_predictor.h}

\subsubsection*{2-bit global history BHT=4}
\inputminted[linenos,fontsize=\scriptsize,frame=leftline]{cpp}{files/src-C2-hybrid_2bit_GH4_predictor.cpp}
\inputminted[linenos,fontsize=\scriptsize,frame=leftline]{cpp}{files/src-C2-hybrid_2bit_GH4_predictor.h}

\subsubsection*{2-bit gshare}
\inputminted[linenos,fontsize=\scriptsize,frame=leftline]{cpp}{files/src-C2-hybrid_2bit_GS_predictor.cpp}
\inputminted[linenos,fontsize=\scriptsize,frame=leftline]{cpp}{files/src-C2-hybrid_2bit_GS_predictor.h}

\subsubsection*{global history BHT=2,4}
\inputminted[linenos,fontsize=\scriptsize,frame=leftline]{cpp}{files/src-C2-hybrid_GH2_GH4_predictor.cpp}
\inputminted[linenos,fontsize=\scriptsize,frame=leftline]{cpp}{files/src-C2-hybrid_GH2_GH4_predictor.h}

\subsubsection*{gshare global history BHT=2}
\inputminted[linenos,fontsize=\scriptsize,frame=leftline]{cpp}{files/src-C2-hybrid_GS_GH2_predictor.cpp}
\inputminted[linenos,fontsize=\scriptsize,frame=leftline]{cpp}{files/src-C2-hybrid_GS_GH2_predictor.h}

\subsubsection*{gshare global history BHT=4}
\inputminted[linenos,fontsize=\scriptsize,frame=leftline]{cpp}{files/src-C2-hybrid_GS_GH4_predictor.cpp}
\inputminted[linenos,fontsize=\scriptsize,frame=leftline]{cpp}{files/src-C2-hybrid_GS_GH4_predictor.h}


%}}}

\end{document}
